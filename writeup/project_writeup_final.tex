\documentclass{article}
%% should indent first paragraph in all sections for consistency
\usepackage{indentfirst}
\usepackage{hyperref}
\hypersetup{
    colorlinks=true,
    linkcolor=blue,
    filecolor=magenta,      
    urlcolor=cyan,
}
%% nice rendering of our system name
\newcommand{\coin}{ \textsc{6824Coin}\ } 

\begin{document}

\title{%
  \coin \\ 
  A Novel Proof-of-Stake Cryptocurrency\\
  \large 6.824 Final Project Writeup}
\author{Henry Aspegren, James Lovejoy \\ $\{ henrya, jlovejoy \}$ @ mit.edu}
\maketitle

\section{Introduction}

Currently the most popular cryptocurrencies use Proof-of-Work. However this consensus mechanism consumes large amounts of energy: Bitcoin alone uses more energy than the entire country of Cuba \cite{energy}. Proof-of-Stake has been proposed as an energy-efficient alternative. In Proof-of-Stake, a single participant - the leader - is selected to add a block to the chain. Leaders are selected probabilistically in proportion to how much of the currency the leader has "staked". Ideally this "stake" constitutes a bond that the leader will lose if she "misbehaves." Realizing Proof-of-Stake in practice has been challenging. Indeed at the time of this writing, no Proof-of-Stake currency has ever achieved widespread adoption\footnote{As defined by being among the top 5 largest currencies by market cap}. 

One of the main reasons for this is the \textit{Nothing-at-Stake} problem. In Proof-of-Work, proposing blocks to extend multiple, conflicting chains requires the consumption of valuable computational power and ultimately any work done on a chain rejected by the network is wasted. This creates a strong incentive for a rational to mine on only one chain - the chain that the network will eventually accept. However in existing Proof-of-Stake designs, the cost of producing a block is negligible. This implies that there is no computational mechanism to prevent a leader from equivocating and extending multiple competing chains. In fact this is may be the rational choice. The \textit{Nothing-at-Stake} problem presents a serious security issue because the network may lose consensus on the 'correct' chain. In this work we propose a novel Proof-of-Stake consensus protocol that solves this problem by cryptographically preventing equivocating. In our protocol, the leader can only extend the chain once and must consume a valuable, although non-computational resource to do so.

\section{Related Work}

\subsection{PeerCoin}
PeerCoin \cite{peercoin} was the first attempt at a Proof-of-Stake cryptocurrency. \coin uses a consensus mechanism inspired by PeerCoin's original design. However because of the \textit{Nothing-at-Stake} problem PeerCoin uses a centralized "check-pointing" scheme. This is a major security concern since a corrupted check-pointing system is a single point of failure for the entire system.  

\subsection{Algorand}
Algorand \cite{algorand} is another alternative to Proof-of-Work. However Algorand, as described by its authors, is not Proof-of-Stake and has a different security argument entirely. The security of Algorand depends on a Byzantine Agreement protocol that is run among a randomly selected set of participants. These participants are selected according to their coins, but there is no concept of a 'stake'. \coin is different from Algorand since it uses Proof-of-Stake and does not use a Byzantine Agreement protocol to achieve consensus.  

\section{\coin Consensus}

In \coin, the consensus algorithm must determine the following:
\begin{enumerate}
\item{Who can propose the next block?}
\item{Which chain do I follow?}
\end{enumerate}
As in Proof-of-Work, the process of adding a block by extending an existing chain performs these two roles. Since the network is decentralized - each participant makes decisions individually. However the entire network will maintain consensus under certain assumptions about the behavior of the individual participants. A sketch of the security argument for \coin will be presented in \ref{security}.

\subsection{Joining the Stake Pool}

To participate in the consensus algorithm, the participant (\textit{staker}) must create a special transaction that \textit{stakes} coins. This \textit{stake} constitutes a bond to the network that will be lost if the \textit{staker} equivocates and produces a block that extends multiple chains. The staking transaction consists of following:
\begin{center}
\textit{stake} = $\{value, R\_point\}$
\end{center}
The $value$ is the total number of coins staked and the $R\_Point$ is a cryptographic commitment. This cryptographic commitment has the useful property that if the \textit{staker} proposes a block for more than one chain it is possible to compute his private key. With knowledge of the private key, anyone can spend his coins. So if the \textit{staker} equivocates, he risks losing his stake entirely.

\subsection{Selecting a Block Proposer}
In Proof-of-Work, a miner (block proposer) can be anyone who can find a nonce such that 
$$SHA256(block\_id || nonce) < target\_difficulty$$
where $||$ is bitwise concatenation. The $target\_difficulty$ is a adjusted by the network as miners come and go to keep the rate of block generation relatively stable. 
In \coin, a block proposer is selected according to their \textit{stake} rather than the ability to find a nonce. Instead a block proposer must meet the following criteria:  
$$SHA256(block\_id || stake || current\_time) < target\_difficulty * (value \times age) $$ 
where $value$ is the coin value of the \textit{stake} transaction and $age$ is the number of blocks that have passed since the \textit{stake} transaction was created. Note that there is no nonce involved and that the computation required is finite: it takes exactly one $SHA256$ hash per output per second to participate in the consensus protocol. By comparison, commercial Bitcoin miners easily churn through over 4 Billion $SHA256$ hashes per second, consuming orders of magnitude more energy. Additionally note that the older and more valuable the \textit{stake} transaction, the easier it is to propose a block. The calculation of $target\_difficulty$ in \coin is identical. 

\subsection{Reaching Consensus}
In Proof-of-Work, the network achieves consensus on which chain by following the chain with the most 'total work'. 'Total work' in this context is the total amount of computational resources devoted to producing the chain. This can be calculated by adding together the $target\_difficulty$ for each block in the chain. In \coin, the network also follows with the chain with the most 'total work', which is also calculated by adding the target difficulty for each block in the chain. However in \coin this reflects the \textit{stake}s consumed to produce that chain not the computational work put into generating the chain. In other words the consensus algorithm follows the chain that was created using the oldest and most valuable outputs.  

\subsection{Preventing Equivocation}
@JAMES HERE 

\section{Implementation}
To implement \coin we used CryptoKernel \cite{ck}, a software package being developed by the Digital Currency Initiative for prototyping cryptocurrencies \footnote{James Lovejoy is the lead developer of CryptoKernel}. CryptoKernel provides out-of-the-box networking and storage code while allowing developers to implement their own consensus modules. We implemented \coin by creating a consensus module in CryptoKernel.

We have created a genesis block for \coin and are currently running several nodes that are all participating in the consensus process. Anyone can join the network and start using the protocol, using our reference implementation on Github \cite{6824coincode}. To join the network simply download the code in the repository and follow the instructions in the README.

Because \coin selects block proposers based on the amount of stake a user has, we have had to think carefully about how distribute the initial coins. In our prototype there are no block rewards and all of the coins were initially given to the Authors. However we have implemented a faucet where you can post your public key and we will send you coins. Our goal is to distribute the vast majority of the coins to anyone who would like them. The faucet can be found at \url{cryptokernel.org/6824coinfaucet}.

One concern is that since initially we will have the vast majority of all coins, we will almost always get to propose the next blocks. In the short term we will avoid this situation by simply not staking some our our coins to deliberately reduce our chances of producing the next block. As others join the network using the faucet this will gradually become a non-issue. 

\section{Analysis}
\large{ \color{red}  { \textbf{ WARNING the authors accept no responsibility for the security of \coin, and it may be insecure and you may lose all of your money } } } 

Analyzing the security of cryptocurrencies is extremely difficult. A full analysis considers both the security of the underlying protocol as well as the game-theoretic implications of rational participants. Actual implementations almost always suffer from unforeseen vulnerabilities. A rigorous analysis is out of the scope of this paper. However we will present a sketch of why we think \coin might be secure.

\subsection{Security Argument (Sketch)}\label{security}

The security of \coin rests on the assumption that participants will choose to extend the chain with the most total work. Because of our commitment scheme a participant can only safely chose to use a given staked output once. If participants equivocate and try to re-use the staked output by extending multiple chains, then they will lose their stake entirely. If participants chose to extend some other chain instead of extending the longest chain the implication is more subtle. The participant will not lose his stake, but rather will lose the ability to propose a block using that stake. This is sub-optimal since the participant will lose our on any rewards associated with producing a block (in \coin, this is the transaction fees). Thus participants have a strong incentive to stake on the chain that the network will ultimately accept, since otherwise their opportunity to stake will be wasted. This is conceptually similar to the argument that a miner in Proof-of-Work will only mine on the longest chain. 

\section{Conclusion}
In the cryptocurrency community, Proof-of-Stake is a hotly debated alternative to energy-intensive Proof-of-Work. However, to the best of our knowledge no one has been able to realize Proof-of-Stake in practice. Furthermore, despite the interest there are few documented protocols and, to the best of our knowledge, no open source implementations of Proof-of-Stake. This work presents a clear description of a Proof-of-Stake algorithm that solves the \textit{Nothing-at-Stake} problem and provides an open source implementation that anyone can use. Our hope is that this work will help the community continue to debate and experiment with Proof-of-Work alternatives.    


\begin{thebibliography}{9}
\bibitem{energy}
http://bigthink.com/strange-maps/bitcoin-consumes-more-energy-than-159-individual-countries
\bibitem{peercoin}
https://peercoin.net/
\bibitem{algorand}
Gilad, Yossi, et al. "Algorand: Scaling byzantine agreements for cryptocurrencies." Proceedings of the 26th Symposium on Operating Systems Principles. ACM, 2017.
\bibitem{ck}
https://dci.mit.edu/cryptokernel/
\bibitem{6824coincode}
https://github.com/henryaspegren/6.824-Final-Project
\end{thebibliography}

\end{document}
